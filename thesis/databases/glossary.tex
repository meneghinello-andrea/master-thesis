%---------------------------------------------------------------------------------------------------
%	glossary.tex
%
% This is the glossary database file. Users must write here the term definitions before using them
% in the document.
%
% Author:  Andrea Meneghinello
% Version: 0.1
% Table of changes:
%		15/03/2016 -> document definition
%---------------------------------------------------------------------------------------------------
\newglossaryentry{deployment-process}
{
	name = {Deployment process},
	plural = {Deployment processes},
	description = {Software deployment include all the activities that make a software system available
		for use. The general deployment process consists of several interrelated activities with possible
		transitions between them.}
}

\newglossaryentry{bug}
{
	name = {bug},
	plural = {bugs},
	description={A software bug is an error, flaw, failure or fault in a computer program or system that
		 causes it to produce an incorrect or unexpected result, or to behave in unintended ways}
}

\newglossaryentry{pay-per-use}
{
	name = {pay-per-use},
	plural = {pay-per-use},
	description = {Metered services is any type of payment structure in which
		a customer has access to potentially unlimited resources but only pays for what they actually use.
		Metered services are becoming increasingly common in enterprise \acs{it} environments. With utility
		computing, for example, a company can purchase computation resources to  match fluctuating needs.
		This approach is promoted as being more cost-effective for the company than maintaining a large
		infrastructure that exceeds the company's average computing power requirements}
}

\newglossaryentry{cloudInfrastructure}
{
	name = {Cloud Infrastructure},
	plural = {Cloud infrastructures},
	description={A Cloud Infrastructure is the collection of hardware and
		software that enables the five essential characteristics of Cloud Computing. The Cloud Infrastructure
		can be seen as containing both a physical layer and an abstraction layer. The physical layer consists
		of the hardware resources necessary to support the cloud services being provided, and typically
		includes server, storage and network components. The abstraction layer consists of software deployed
		across the physical layer, which manifests the essential Cloud characteristics. Conceptually the
		abstraction layer sits above the physical layer}
}

\newglossaryentry{middleware}
{
	name = {Middleware},
	plural = {Middleware},
	description = {Middleware is a set of software that act as brokers between infrastructures and software,
		allowing communication despite different communication protocols or \acs{os}}
}

\newglossaryentry{agile}
{
	name = {Agile software development},
	description = {agile software development is a group of software development methodologies based on
		incremental development}
}

\newglossaryentry{sla}
{
	name = {\acf{sla}},
	plural = {Service Level Agreements},
	description = {A \acf{sla} is a part of a standardized service contract where a service is formally defined.
		Particular aspects of the service – scope, quality, responsibilities – are agreed between the service provider
		and the service user. A common feature of an \ac{sla} is a contracted delivery time (of the service or performance)}
}

\newglossaryentry{slo}
{
	name = {\acf{slo}},
	plural = {Service Level Objectives},
	description = {A \acf{slo} is a key element of a \ac{sla} between a service provider and a customer. \ac{slo}s
		are agreed as a means of measuring the performance of the Service Provider and are outlined as a way of avoiding
		disputes between the two parties based on misunderstanding}}