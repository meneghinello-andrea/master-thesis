%---------------------------------------------------------------------------------------------------
%		problem.tex
%
%	This file contains the sections that describe the problem addressed in the thesis
%
%	Author: Andrea Meneghinello
% Version: 0.1
%	Table of changes:
%		11/03/2016 -> document definition
%---------------------------------------------------------------------------------------------------
\section{Problem definition}
\label{sec:background-problem}
Cloud computing is becoming more mature day by day and market has started investing on it. A recent
study, commissioned by Microsoft \cite{microsftCloudNewJob}, shows that cloud computing produces much
new innovation in \acs{it} every year:

\begin{center}
	\begin{quote}
		``Innovation created by cloud computing produce \textdollar{}1.1 trillion a year in new
		business revenues. It was also able to generate 14 million of jobs worldwide from 2011 to 2015.''
	\end{quote}
\end{center}

Therefore the trend is positive and it is expected to grow more and more in the long term.

We have argued about the major requests made by software developers. Their need for automatic
hardware configuration lead to the creation of the \ac{paas} cloud model. \ac{paas} differs from
\ac{iaas} because \ac{paas} does not allow users to control the underlying hardware. Despite all
\ac{paas} advantages (described in Section \ref{sec:background-paas-characteristics})
the following issues may be still present:

\begin{itemize}
	\item{it may lead to a possible vendor lock-in if the provider imposes the use of a particular
		technology;}
	\item{it may compromise cloud resource elasticity because users are not able to configure the
		underlying hardware.}
\end{itemize}

In order for the \ac{paas} vendors to offer high quality services, both for companies and software-houses,
they must provide ready-to-use work environments and key mechanisms that guarantee \keyword{elasticity}
for the deployment units.

Elasticity can be obtained by the combination of two complementary dimensions: \keyword{optimization of 
the \ac{paas} underlying hardware management} and \keyword{specifically designed software
architectures}.

The assets on which the \ac{paas} vendors can make optimizations are those illustrated in Section
\ref{sec:background-virtualization-assets}, hence our purpose is to compare and then analyse
classic virtualization techniques and the ones provided by the Docker framework in terms of computing,
storage and networking.

In addiction we will seek for an architectural pattern that is able to exploit the elasticity mechanisms 
provided by the \ac{paas} layer.