%---------------------------------------------------------------------------------------------------
%		objective.tex
%
%	This is the main file of the chapter that talk about multi-tenancy.
%
%	Author: Andrea Meneghinello
% Version: 0.1
%	Table of changes:
%		12/03/2016 -> document definition
%---------------------------------------------------------------------------------------------------
\section{Objective}
\label{sec:elasticity-objective}
To conclude our discussion about elasticity and multi-tenancy we can assert that offering a multi-tenant
service deployed over a \ac{paas} is a good solution both for companies and software-houses. Both can
earn money due to economy of scale.

As we just argued in this chapter, in order to offer a multi-tenant service, the application must be
deployed over a \ac{paas} layer that is able to manage in an efficient way the elasticity in order to
support different workloads. Two complementary dimensions must be considered:

\begin{itemize}
	\item{elasticity of the \ac{paas} underlying hardware assets. Hence the software-houses have to
		choose the one that offer a virtualization layer that is able to exploit these assets as best it
		can;}
	\item{design a good software architecture. A monolithic architecture is not the best option even if
		placed over a good \ac{paas} provider.}
\end{itemize}

The purpose of the next chapters of this thesis is to analyse these two complementary aspects. In particular
the next chapter will illustrate some comparison tests that are been executed in order to compare how two
different virtualization technologies are able to exploit the assets illustrated in Section
\ref{sec:background-virtualization-assets}. Instead in the last chapter we want to propose a possible
software architecture that exploits the assets offered by \ac{paas} provider as best as it can
in order to obtain a multi-tenant deployed service.