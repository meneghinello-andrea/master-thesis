%---------------------------------------------------------------------------------------------------
%		introduction.tex
%
%	This is the main file of the chapter that talk about elasticity.
%
%	Author: Andrea Meneghinello
% Version: 0.1
%	Table of changes:
%		17/03/2016 -> document definition
%---------------------------------------------------------------------------------------------------
\section{Introduction}
\label{sec:elasticity-introduction}
With the marked attention posed on cloud computing and its swiftly risen in the last years, we had
an immediate perception of the value added. We have seen marked progress in the development of
\ac{saas} and \ac{iaas} layers leaving the \ac{paas} one unexplored.

Nevertheless, nowadays there are reasons to consider it an emerging business, especially attractive
for \ac{saas} providers. As we argued in Chapter \ref{cap:elasticity}, a key task of this layer is
to manage the elasticity issues, but alone, the \ac{paas} layer, is not able to accomplish to this owe.

\ac{saas} providers are seeking for automation solutions that help them in \keyword{building},
\keyword{deploying} and \keyword{managing} their applications in the most flexible as possible way. A
platform offered as an abstraction layer that hides the complexity of the underlying infrastructures
is very attractive to that endeavour. In abstract, the scope of \ac{paas} is to mediate interests
between \ac{saas} and \ac{iaas} layers. Essentially, we want to avoid over-provisioning of resources
but we also want some policies to preserve the \ac{sla} stipulated with end-users. This mediation
of interests can be understood as \keyword{system elastiticy}.

Nowadays, given the absence of a fully standard \ac{paas} solution the other two layers of the \ac{spi}
model (\ac{saas} and \ac{iaas}) have tried to provide mechanisms to manage elasticity, \ac{qos} and the
application life-cycle. However, both of them have characteristics that lead us to consider that they are
not the right ``place'' to manage them, because:

\begin{itemize}
	\item{in the \ac{iaas} layer there is an evident conflict of interests. On one hand we do not want
		to manage the hardware infrastructure, on the other we are forced to manage it because this layer
		provide to us hardware resources in a on-demand way;}
	\item{instead, in the \ac{saas} we want to keep low coupling with the infrastructure. In this player
		we have a much high-level view, so we are not interested on how the physical resources
		are managed.}
\end{itemize}

Thus, the need of a mediator is risen, and \ac{paas} layer is the right place to address these challenges
and the related complexity.

According to the \ac{nist} definition (given in Section
\ref{sec:background-cloudComputing-cloudServiceModels}), \ac{paas} gives freedom to the customer regarding
application reconfiguration, since end-users should only deal with the general ``configuration settings
for the application-hosting environment'' but do not need to worry about the mechanism being needed
for implementing those reconfigurations. Reconfiguration management is the responsibility of \ac{paas}
providers. Since every cloud service should be elastic (accordingly with cloud definition), this mean
that \ac{paas} providers should deal with many mechanism that automate the service scalability and
adaptability.

However, the level of automation needed to approach a cost-optimal exploitation of the service is still
changing nowadays because of the many aspects that should be considered. On one hand, scalability
decisions must match what has been stated in the \ac{sla}. This means that those decisions should be
taken as soon as the workload or the service performance starts to vary, which strongly suggests the
need to include some workload prediction mechanism. Nevertheless, forecasting techniques are not perfect
hence, they should be complemented with other reactive mechanism; e.g. when the resulting service
performance level do not comply with what is being specified in the \ac{sla} or lead unnecessary
over-provisioning costs, service providers should take appropriate actions. Those actions may consist
in adding service instances or migrating those instances to better \ac{vm}s when service capacity
should be increased, or in the opposite case, in releasing instances when service capacity need to be
decreased.

One of the regular dimensions in \ac{sla}s is availability. Successful distributed services are
concurrently used by many customers. Service providers should guarantee service continuity, otherwise
service clients will not rely on that provider and they will look for another, more reliable, one.
Unfortunately, since complex software systems are always in need of modification, both platform and
service components need to be eventually updated in order to fix bugs, remove security vulnerabilities
or enhance their functionality. This software upgrading process might put at risk the availability levels
specified in a \ac{sla}. So, this is another source of trouble for service providers.

