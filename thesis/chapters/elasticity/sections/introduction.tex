%---------------------------------------------------------------------------------------------------
%		introduction.tex
%
%	This is the main file of the chapter that talk about elasticity.
%
%	Author: Andrea Meneghinello
% Version: 0.1
%	Table of changes:
%		17/03/2016 -> document definition
%---------------------------------------------------------------------------------------------------
\section{Introduction}
\label{sec:elasticity-introduction}
A requirement for distributed services is availability: they must remain regularly available to be
used by as many customers as needed with \keyword{acceptable performance} and \keyword{scalability}.
Distributed services and systems have always been required to be scalable. Replication of services
components, in order to balance the workload arriving to each replica, has been the chosen approach.

Services needed to be carefully designed in order to minimize synchronization needs among their
components with the aim of not blocking their execution. Evidently, unlimited scalability could not be
achieved since it requires infinite infrastructure resources. So, traditionally, scalability of a
service depends on its software architecture being limited by the amount of hardware resources secured
in the data-centre where it was deployed.

As we have seen in the previous chapter, the advent of cloud computing partially broke those limits, but
elasticity management remains not trivial.

According to the \ac{nist} definition (given in Section
\ref{sec:background-cloudComputing-cloudServiceModels}), \ac{paas} gives freedom to the customer regarding
application reconfiguration, since end-users should only deal with the general ``configuration settings
for the application-hosting environment'' but do not need to worry about the mechanism being needed
for implementing those reconfigurations. Reconfiguration management is the responsibility of \ac{paas}
providers. Since every cloud service should be elastic (accordingly with cloud definition), this mean
that \ac{paas} providers should deal with many mechanism that automate the service scalability and
adaptability.

However, the level of automation needed to approach a cost-optimal exploitation of the service is still
changing nowadays because of the many aspects that should be considered. On one hand, scalability
decisions must match what has been stated in the \ac{sla}. This means that those decisions should be
taken as soon as the workload or the service performance starts to vary, which strongly suggests the
need to include some workload prediction mechanism. Nevertheless, forecasting techniques are not perfect
hence, they should be complemented with other reactive mechanism; e.g. when the resulting service
performance level do not comply with what is being specified in the \ac{sla} or lead unnecessary
over-provisioning costs, service providers should take appropriate actions. Those actions may consist
in adding service instances or migrating those instances to better \ac{vm}s when service capacity
should be increased, or in the opposite case, in releasing instances when service capacity need to be
decreased.

One of the regular dimensions in \ac{sla}s is availability. Successful distributed services are
concurrently used by many customers. Service providers should guarantee service continuity, otherwise
service clients will not rely on that provider and they will look for another, more reliable, one.
Unfortunately, since complex software systems are always in need of modification, both platform and
service components need to be eventually updated in order to fix bugs, remove security vulnerabilities
or enhance their functionality. This software upgrading process might endanger the availability levels
specified in a \ac{sla}. So, this is another source of trouble for service providers.

