%---------------------------------------------------------------------------------------------------
%		summary.tex
%
%	This is file contains the chapter summary.
%
%	Author: Andrea Meneghinello
% Version: 0.1
%	Table of changes:
%		21/03/2016 -> document definition
%---------------------------------------------------------------------------------------------------
\section{Summary}
\label{sec:measurements-summary}
After the execution of the tests and have analysed the collected results we can now derive some conclusion
about the two virtualization technologies: \ac{kvm} \ac{vm}s and Docker containers. 

Both are, nowadays, mature technologies that have benefited of incremental hardware and software updates.
In general, we can assert that there is, at most, a point of view that is advantageous for the hardware-virtualization
level technologies and others that are more appropriate for the \acs{os}-virtualization level ones.

If we look at the hardware performance we can observe that the Docker containers are able to exploit very well
the available underlying hardware, except for the network delay (see Section \ref{sec:measurements-network-result}).
Instead, in matter of adapt themselves to share the underlying available resources, we found that the
\ac{vm}s are more mature than the counterpart.

After working a little bit with Docker technology, we have appreciated its lightweight, the rapidity that
it follows, and the easy \acs{api} that they provide to developers. Even though the Docker container are more
recent than the \ac{kvm} \ac{vm}s, they provide to developers a good level of virtualization and make some tasks
more easier (like build environments.) We expect that in nearly future this technology will be improved and largely
preferred by many developers. 

As we asserted in the previous chapter, the only adoption of a good virtualization layer, that is able to exploit
the underlying hardware, is not sufficient to guarantee good levels of elasticity. Because of this, in the next
chapter, aware of the experience matured with the performed tests, we want to introduce a possible software
architecture that can provide good level of elasticity combined with the \acs{os}-virtualization level and that
lead developers to easily generate multi-tenants applications. 