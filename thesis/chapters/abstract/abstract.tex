%---------------------------------------------------------------------------------------------------
%		abstract.tex
%
%	This is the main file of the chapter that talk about the problem space.
%
%	Author: Andrea Meneghinello
% Version: 0.1
%	Table of changes:
%		15/03/2016 -> document definition
%---------------------------------------------------------------------------------------------------
\begin{abstract}
	The attention for cloud computing and the take up of its offering have risen swiftly over the last
	decade. The cloud computing stack, also known as \acs{spi}, spans three conceptual layers with
	associated service models: \acs{saas} on the top (at user end), \acs{paas} in the middle (at
	developers end) and finally \acs{iaas} on the bottom (at sysadmin end) where physical resources
	reside.
		
	Owing to more immediate perception of value added, the cloud landscape has shown marked progress
	in the development of the \acs{saas} and \acs{iaas} offerings. Interest in \acs{paas} solutions
	have taken longer to ignite instead, so much so that the \acs{paas} world is still largely unexplored.
		
	The purpose of this thesis is to explore the \acs{paas} layer in order to understand how it is
	different from the \ac{iaas} layer and how we are able to build elastic applications upon it.
	
	We analyse in deep, through the evaluation of infrastructural costs, two units of deployment for
	the \ac{paas} (\acs{vm}s and containerization) in order to choose the better one to host our service.
	Finally, we want to propose a possible software architecture that, through micro-services, is able
	to exploit the elasticity mechanisms provided by the platform.
\end{abstract}