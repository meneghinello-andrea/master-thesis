%---------------------------------------------------------------------------------------------------
%		introduction.tex
%
%	This is file contains an introduction that lead us to consider SOA as a possible architecture
% to embrace elasticity and multi-tenancy.
%
%	Author: Andrea Meneghinello
% Version: 0.1
%	Table of changes:
%		21/03/2016 -> document definition
%---------------------------------------------------------------------------------------------------
\section{Introduction}
\label{sec:architecture-introduction}
In the previous chapter we have seen how two different deployment units (\ac{kvm} \ac{vm}s and Docker
containers) are able to exploit the computing assets. Those tests lead us to affirm that even if
the Docker containers are less adaptable, when they share the same hardware resources (see \acs{cpu}
and storage benchmark tests in Section \ref{sec:measurements-cpu-results} and
\ref{sec:measurements-storage-results} respectively), they can fit best to reach the elasticity
requirements illustrated in Chapter \ref{cap:elasticity}.

This happens because, using the same workload \ac{vm}s reach the scale-out points before Docker
containers thus \ac{vm}s' behaviour, during the execution of scale-in/out operations (see Section
\ref{sec:elasticity-requirements-scalability}), can compromise some of the \ac{sla} requirements
(i.e. perceived \ac{qos}). Instead, given the lightweight nature of Docker containers, when they reach
the scale-out/in points the associated operations can be executed in less time avoiding the probability
to break the \ac{sla} requirements with the end-users. Therefore, it is reasonable if \ac{paas} providers
base their virtualization technologies over containers, as the \ac{paas} vendors in Section
\ref{sec:background-paas-platforms} already did.

When a developer decides to deploy an application over the \ac{paas} service described above, it is
making it available anywhere and accessible with any type of device, so the application becomes a
distributed service. A requirement for distributed services is  availability: they must remain regularly
available to be used by as many customers as needed with \keyword{acceptable performance} and \keyword{scalability}.
Distributed services and systems have always been required to be scalable. Replication of services components,
in order to balance the workload arriving to each replicas, has been the chosen approach.

Services needed to be carefully designed in order to minimize synchronization needs among their components
with the aim of avoiding the block their execution and offering high levels of \ac{qos}. So, the
developers must exploit the elasticity properties offered by the \ac{paas} vendors through a proper
designed software architecture.

Actually, applications built with the classic principles of the software engineering produce output 
that we can classify with the term ``monolithic''. Most of the time, applications are made-up of a
single executable file that contains all the software functionalities. Thus, they do not suit the cloud
paradigm because they are too coarse-grained to scale efficiently. Thus, a need for an architectural 
design that is able to efficiently scale our applications or services is risen.

An architectural pattern that can help us reaching the target is \acf{soa}. This pattern was born in 
pre-cloud era, but it has many common points with the cloud paradigm. Through its adjustment we are
able to exploit its benefits (like \keyword{flexibility} and \keyword{agility}) and we can make
it express the cloud property of elasticity.