%---------------------------------------------------------------------------------------------------
%		introduction.tex
%
%	This is file contains an introduction that lead us to consider SOA as a possible architecture
% to embrace elasticity and multi-tenancy.
%
%	Author: Andrea Meneghinello
% Version: 0.1
%	Table of changes:
%		21/03/2016 -> document definition
%---------------------------------------------------------------------------------------------------
\section{Introduction}
\label{sec:architecture-introduction}
With the marked attention posed on cloud computing and its swiftly risen in the last years, we had
an immediate perception of the value added. We have seen marked progress in the development of
\ac{saas} and \ac{iaas} layers leaving the \ac{paas} one unexplored.

Nevertheless, nowadays there are reasons to consider it an emerging business, especially attractive
for \ac{saas} providers. As we argued in Chapter \ref{cap:elasticity}, a key task of this layer is
to manage the elasticity issues but alone, the \ac{paas} layer, is not able to accomplish to this owe.

\ac{saas} providers are seeking for automation solutions that help them in \keyword{build},
\keyword{deploy} and \keyword{manage} their applications in a most flexible as possible way. A
platform offered as an abstraction layer that hides the complexity of the underlying infrastructures
is very attractive to that endeavour. In abstract, the scope of \ac{paas} is to mediate interests
between \ac{saas} and \ac{iaas} layers. Essentially, we want to avoid over-provisioning of resources
but we also want some policies to preserve the \ac{sla} stipulated with end-users. This mediation
of interests can be understood as \keyword{system elastiticy}.

Nowadays, given the absence of a fully standard \ac{paas} solution the other two layers of the \ac{spi}
model (\ac{saas} and \ac{iaas}) have tried to provide mechanisms to manage elasticity, \ac{qos} and the
application life-cycle. However, both of them have characteristics that lead us to consider that they are
not the right ``place'' to manage them, because:

\begin{itemize}
	\item{in the \ac{iaas} layer there is an evident conflict of interests. On one hand we do not want
		to manage the hardware infrastructure, on the other we are forced to manage it because this layer
		provide to us hardware resources in a on-demand way;}
	\item{instead, in the \ac{saas} we want to keep low coupling with the infrastructure as possible. In
		this layer we have a much high-level view, so we are not interested on how the physical resources
		are managed.}
\end{itemize}

Thus, the need of a mediator is risen, and \ac{paas} layer is the right place to address these challenges
and the related complexity.

While we have understood the potentiality for the \ac{paas} layer we  still haven't apprehended why
classic architectural patterns are not ``correct'' for the cloud computing model. Applications built
with the classic principles of the software engineering produce output that we can classify with the
term ``monolithic''. Most of the time, applications are made-up of a single executable file that contains
all the software functionalities. Thus, they do not suit the cloud paradigm because they are too
coarse-grained to scale efficiently. Thus, a need for an architectural design that is able to efficiently
scale our applications or services is risen.

An architectural pattern that can help us reaching the target exists, and its name is \acf{soa}. It is
a pattern born in pre-cloud era, that revisited, bring some benefits such as: \keyword{flexibility} and
\keyword{agility}. In the following section we are going to analyse this architectural design.