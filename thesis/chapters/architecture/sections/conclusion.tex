%---------------------------------------------------------------------------------------------------
%		conclusion.tex
%
%	This is file contains an introduction about the SOA architectural desing.
%
%	Author: Andrea Meneghinello
% Version: 0.1
%	Table of changes:
%		21/03/2016 -> document definition
%---------------------------------------------------------------------------------------------------
\section{Conclusion}
\label{sec:architecture-conclusion}
In this thesis we tried to propose a possible a solution to simplify the necessary efforts to build
new software by developers.

As we argued in Section \ref{sec:background-problem} business companies do not want to manage the
necessary hardware to keep the application running according to the \ac{sla} requirements. Because
they will focus their efforts in the business and want to be supported by the software.

In the same section we also argued about the requests made by software-house which want to focus
their efforts in build and deploy software and not the environments to support it.

In this thesis we have learned that the \ac{paas} cloud model recently is becoming very interesting,
and in the nearly future the attention on it will increases again. This service model can answer to
both the set of requests made by companies and software-houses.

We have seen some basic capabilities offered by Docker containers and we have understood how they are
changing the \ac{paas} market. Many \ac{paas} providers in the market starting a migration to this
framework because it offers many useful functionalities and simplify the deployment model (see Section
\ref{sec:background-paas-platforms}). Even if we have seen, by the benchmark results in Chapter
\ref{cap:measurements}, that Docker container are less good in sharing the underlying hardware
assets, they are good to build elastic and multi-tenant service over the \ac{paas} cloud model
(see Chapter \ref{cap:architecture}).

With Docker we have available a powerful platform agnostic tool that permits to us to simply deploy
our applications and port them easily to different environments. With Docker, for software-houses
is easy to build home-made test environments when the software is under test and then deploy it in
a production environment. The only necessary thing is that both the environments have a running
Docker daemon.