%---------------------------------------------------------------------------------------------------
%		problem.tex
%
%	This file contains the sections that describe the problem addressed in the thesis
%
%	Author: Andrea Meneghinello
% Version: 0.1
%	Table of changes:
%		11/03/2016 -> document definition
%---------------------------------------------------------------------------------------------------
\section{Problem definition}
\label{sec:problemSpace-problem}
Cloud computing is becoming more mature day by day and market has started investing on it. A recent
study, commissioned by Microsoft \cite{microsftCloudNewJob}, shows that cloud computing produces much
new innovation in \acs{it} every year:

\begin{center}
	\begin{quote}
		``Innovation created by cloud computing produce \textdollar{}1.1 trillion a year in new
		business revenues. It was also able to generate 14 million of jobs worldwide from 2011 to 2015.''
	\end{quote}
\end{center}

\ac{paas} differs from \ac{iaas} because \ac{paas} does not allow users to control the underlying
hardware. Despite all \ac{paas} advantages (described in Section \ref{sec:problemSpace-paas-paasCharacteristics})
the following issues may be still present:

\begin{itemize}
	\item{it may lead to a possible vendor lock-in if the provider imposes the use of a particular
		technology;}
	\item{it may compromise cloud resource elasticity because users are not able to configure the
		underlying hardware.}
\end{itemize}

In order for the \ac{paas} vendors to offer developers high quality services, they should use a 
virtualization layer that exploits the underlying hardware as best it can. The assets on which the
virtualization layer can make optimizations are those illustrated in Section
\ref{sec:problemSpace-virtualization-assets}, hence our purpose is to compare and then analyse
classic virtualization techniques and the ones provided by the Docker framework in terms of computing,
storage and networking.

In addiction we want to find a way to obtain elasticity, without considering hardware level. Thus new
architectural patterns must be found in order to exploit elasticity offered by the cloud paradigm.

Finally we want to understand the portability value that Docker offers in order to overcome the vendor
lock-in problem.